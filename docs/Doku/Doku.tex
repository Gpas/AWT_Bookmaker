\listfiles
\documentclass[a4paper, abstracton]{scrartcl}
\usepackage[utf8]{inputenc}
\usepackage{amsmath, amsfonts, enumerate}
\usepackage{graphicx}		% graphics
\usepackage[colorinlistoftodos]{todonotes}
\usepackage{hyperref}		% enables the usage of links
\usepackage{glossaries}		% glossary
\setlength{\parindent}{0pt} 	% disable padding of first sentence in a paragraph
\setlength{\parskip}{11pt}		% disable gap between parapgraphs
\usepackage{fixltx2e}		% enables textsubscripting
\usepackage{fancyhdr}		% used for page layout
\usepackage{caption}
\usepackage{subcaption}
\usepackage{float}
\usepackage{listings}		% http://ctan.org/pkg/listings

\lstset{
  basicstyle=\ttfamily,
  mathescape
}

% add font of scrartcl / koma: -> european writing style, wider pages and so on. See def. of KOMA / scratcl
\addtokomafont{author}{\normalsize} 
\addtokomafont{date}{\normalsize} 
%\setkomafont{disposition}{\bfseries}

\setlength{\headheight}{48pt}

% style of all other pages
\fancypagestyle{basicpagestyle}{
	\fancyhf{}								% clear everything
	\fancyfoot[C]{\thepage}
	\renewcommand{\footrulewidth} {1pt}		% set rule thickness
	\renewcommand{\headrulewidth} {1pt}
	\rhead{Bern University of Applied Sciences}
	\lhead{\includegraphics[width=0.8cm]{images/BFHlogo.png}}
	% \fancyfoot[L]{}
	% \fancyfoot[R]{}

}

% style of first page
\fancypagestyle{titlepage}{
   \fancyhf{}
   \renewcommand{\footrulewidth} {0pt}		% set rule thickness
   \renewcommand{\headrulewidth} {1pt}
   \lhead{\includegraphics[width=0.8cm]{images/BFHlogo.png}} %  add logo
   \fancyhead[R]{Bern University of Applied Sciences} % text
}

\pagestyle{basicpagestyle}

\begin{document}
\title{\vspace{1cm}AWT Bookmaker \vspace{1cm}}
\author{
  Stefan Andonie\\
  \texttt{andos1@students.bfh.ch}
  \and
  Pascal Grüter\\
  \texttt{grutp1@students.bfh.ch}\vspace{1cm}
}

\date{Biel, \today}
\vspace{3cm}
\maketitle
\thispagestyle{titlepage}

%\includegraphics[width=15cm]{images/titlepage.jpeg}

\pagebreak

\section{Einleitung}

  Für das Semesterprojekt im Modul Advanced Web Technologie wurde ein Wettsystem
  als Webapplikation implementiert.
  Das System wurde mit Java Server Faces realisiert und soll auf einem Apache
  Tomcat Applikationsserver laufen.
  
  Das Wettsystem soll es einem Buchmacher erlauben Fussballspiele zu erfassen
  und Wettzenarien für gewisse Ergebnisse und Ereignisse währen eines Spiels mit
  dazugehörigen Wettquoten zu definieren.
  Danach sollen Spieler Wetten darauf abschliessen können.

\section{Architektur}

  Unsere Implementierung folgt dem Model, View, Controller pattern.
  Die Views wurden wie üblich mit JSF in den xhtml files definiert.
  Diese Views werden jeweils von einem Java Bean gesteuert, das ist der
  Controller Teil. Die Buissneslogik befindet sich getrennt in eigenen Java
  Klassen die unser Domainmodel abbilden und auf welche die Controller zugreifen.
  
  Die Daten bzw. die instanzen der Domainclasses werden mit Hilfe eines ORMappers
  an eine Relationelle Datenbank angebunden und so persistiert.

  \subsection{Projektstrucktur}

  Dieses Patten findet sich in der Strucktur des Projektes wider.
  Im Sourceverzeichniss finden wir das bookmaker und das model packet und
  ausserhalb den das Webverzeichniss, es enthält die Views und deren ressourcen
  und die JSF konfiguration.
  Das bookmakerpacket enthält sämtliche Controllers und einige Hilfsklassen und
  Methoden die es erlauben properties und Objekte zu laden oder Objekte zu
  manipulieren und zu speichern.
  Dort werden die Ausgabe werte für die Views berechnet und das System bedient.
  Das Modelpacket enthält die Domainmodel Klassen. Die Darin enthaltene
  Buissneslogik erlaubt es benutzer und Wetten zu erstellen und Wetten
  Abzuschliessen. Hier wird der Zustand des Systems abgebildet.
  Des Weiteren befinden sich im Sourceverzeichniss die Mehrsprachigen properties
  files und weiter konfiguretions files.

\section{Model}
 (Siehe Abbildung \ref{fig:domain_model} auf Seite \pageref{fig:domain_model})

  \begin{figure}[h!]
  \begin{center}
    \includegraphics[width=0.7\textwidth,angle=-90]{images/DomainModel.pdf}
  \end{center}
  \caption{Domain Model}
  \label{fig:domain_model}
\end{figure}

\section{Views und Controller}

\section{Fazit}

\begin{abstract}
Zusammenfassung
\end{abstract}
\pagebreak
\tableofcontents	% add table of contents (takes 2x typesetting to generate)
\pagebreak
\listoffigures		% add list of figures (takes 2x typesetting to generate)

\pagebreak	% see difference between pagebreak, newpage and clearpage


\pagebreak


\pagebreak	


\end{document}
